\documentclass[10pt]{article}
\usepackage[utf8]{inputenc}
\usepackage[T1]{fontenc}
\usepackage[danish]{babel}
\usepackage{amsfonts}
\usepackage{amsmath}
\usepackage[pdftex]{color,graphicx}
\usepackage{graphicx}
\usepackage{wrapfig}
\usepackage{float}

\begin{document}

\section{Dictionary}
[1] KuroseRoss Computer Networking

\begin{description}
\item[API] fork. Application Programming Interface. En grænseflade der
    definerer hvordan man skal benyttes sig af et givent system.
\item[end system] En terminerende ende af netværket. F.eks. Server, desktop
    playstation eller smartphone
\item[hosts] Se ``end system''. Disse kan deles i to kategorier, klienter og
    servere.
\item[network edge] Se ``end system''
\item[communication links] Section 1.2
\item[packet switches] En enhed der modtager på et link og sender ud på et
    eller flere andre. VIGTIGT - disse enheder kører ikke internet
    applikationer(Se \textbf{distributed applications}), de faciliterer kun
    udvekslingen af dat imellem \textbf{end systems}. Typisk enten 
    \textbf{routers} eller \textbf{link-layer switches}. 
    Se ``Bil analogi''(DIKURevy 2012)
\item[router] Typisk i kernen af nettet
\item[link-layer switch] Typisk i adgangs netværk(\textbf{access networks})
\item[access networks]
\item[dial up] en internet forbindelse gennem telefon nettet, hvor der ringes
    op til en \textbf{ISP}s modem, som så forbinder brugeren med nettet. Max
    hastighed 56 kbps
\item[DSL] fork. Digital Subscriber Line. En forbindelse gennem telefon nettet.
    der bruges dog et DSL modem og i \textbf{CO}'en er man forbundet til en 
    \textbf{DSLAM}. Her udnyttes forskellige frekvens bånd til at ``dele''
    forbindelsen. Max hastighed $\downarrow$ 2Mbps $\uparrow$ 1Mbps
\item[cable] en forbindelse der ligner DSL, men er igennem TV udbyderens net
    istedet for tele udbyderens net.
\item[FTTH] fork. Fiber-To-The-Home. Der ligges fiberkabler ud til kunden og de
    modtager både TV, Radio, telefoni og internet gennem denne forbindelse. Der
    findes to typer \textbf{PON}(Passive Optical Network) og \textbf{AON}(Active
    Optical Network).
\item[PON]
\item[AON]
\item[telco] Lokal teleudbyder, typisk udbyderen af kober hvor igennem nettet
    udbydes.
\item[packets] En pakke af information der bliver transmiteret over netværket.
    Bestående af \textbf{packet header} og data.
\item[packet header] BLA BLA
\item[path] en vej igennem nettet
\item[route] se \textbf{path}
\item[transmission rate] $\frac{bits}{sekund}$, forkortes bps(bits per sekund).
\item[ISP] fork. Internet Service Provider. Udbyder af net, Universitets-,
    corporate-, telefoni udbyder. Findes i flere niveauer - tier {1-n}, hvor 1
    er top niveau udbydere der alle er forbundet til hinanden, og tier n er
    udbydere af internet til almindelige jævne mennesker.
\item[protocol] dansk protokol - kontrakt der styrer hvordan data sendes
    og modtages over nettet. En protokol er en mængde handlinger og
    konventioner der faciliterer og muligør udveksling af data. [1] S. 9
\item[CO] fork. Central Office. En bygning/instans hvor en række brugere er
    forbundet.
\item[DSLAM] fork. Digital Subscriber Line Access Multiplexer. Splitter telefon
    og internet trafik fra en husholding og sender det til hhv telefon nettet
    og internettet.
\item[TCP/IP] Se \textbf{TCP} og \textbf{IP}
\item[TCP] fork. Transmission Control Protocol. 
\item[IP] fork. Internet Protocol.
\item[IETF] fork. Internet Engineering Task Force. Organisationen der udsteder
standarderne(Se \textbf{RFC}) for internettet.
\item[RFC] fork. Request For Comment
\item[IEEE] fork Institute of Electrical and Electronics Engineerings. Udsteder
    af \textbf{IEEE 802}
\item[IEEE 802] serie af standarder for \textbf{LAN}/\textbf{MAN}, f.eks.
    Ethernet og WiFi.
\item[LAN] fork. Local Area Network. Et netværk der fungerer på lokalt plan
\item[MAN] fork. Metropolitan Area Network. Et netværk der fungerer på by plan
\item[WAN] fork. Wide Area Network. Et netværk der fungerer på tværs af byer og
    lande.
\item[WWAN] fork. Wireless WAN. Se \textbf{WiFi}, \textbf{3G} og \textbf{WiMAX} 
\item[distributed applications] applikationer der baserer sig på at være
    forbundet. Altså to eller flere \textbf{end systems} der udveksler data.
\item[client program] Et program der sender og modtager data fra et
    \textbf{server program}
\item[server program] Et program der behandler forespørgsler og svarer passende
    på disse.
\item[twisted-pair kober] Typisk hjemmenetværk, også kaldet ethernet kabler.
\item[Coaxial kabel] Typisk det kabel man forbinder antennen til sit fjernsyn
    med, bruges i f.eks. \textbf{cable} internet, imellem væg og modem.
\item[Fiber optisk] Tynde kabler der benytter sig af lys istedet for
    elektroniske signaler. Hastigheder op til 100 Gbps. Ingen elektromagnetisk
    forstyrrelse og næsten ingen signal forringelse over 100 kilometer. Ikke så
    udbredt pg.a. høje omkostninger i udstyret.
\item[switching, circuit] Fremgangsmåde for at flytte data i nettet. Med denne
    reserveres alle resourcer langs pakken \textbf{route} mens der
    kommunikeres.
\item[switching, packet] Fremgangsmåde for at flytte data i nettet. Med denne
    reserveres ingen resourcer, og der kan derfor opstå kø, tilgengæld kan man
    ofte få større udnyttelses af netværket.
\item[FDM] fork. Frequency-Division Multiplexing. Bruges i circuit switching(Se
    \textbf{switching, circuit}), med denne metode dele frekvens båndet op
    imellem forskellige brugere.
\item[TDM] fork. Time-Division Multiplexing. Alternativ til FDM, hvor man
istedet for at give en del af frekvensbåndet, deler tid op i frames, som så
deles op i slots, hvor hvert slot tildeles et sender-modtager par.
\item[store-and-forward] teknik der benyttes når man skal sende pakker gennem
    et netværk, først opbevares pakken og derfter sendes den. Denne metode
    sørger for at et enkelt link kan benyttes af flere brugere samtidigt.
\item[processing delay] tiden det tager at inspicerer \textbf{packet headers}
    og vælge hvor pakken skal sendes hen.
\item[queuing delay] ventetiden der forekommer når en pakke venter på at blive
    sendt via linket.
\item[transmission delay] er den ventetid der forekommer når man venter på at
    alle en pakkes bits er blevet sendt via linket. Defineret som følger:
    $L = \text{antallet af bits i pakken}, R = \text{transmission rate,
    bps}$ og $\text{transmission delay} = \frac{L}{R}$
\item[propagation delay] ventetiden ved overførsel af data, altså tiden det
    tager en mængde bits at rejse over en afstand. Hastigheden varierer som
    regel imellem $2 \dot 10^8 \text{m/s og } 3 \dot 10^8 \text{m/s}$.
    Propagation delayet bliver således $d=\text{afstand}, s=\text{propagation
    speed}, d/s$
\item[total nodal delay] = $d_{proc} + d_{queu} + d_{trans} + d_{prop}$.
\item[end-to-end delay] = $N(d_{proc} + d_{trans} + d_{prop}), N =
    \text{Antallet af hop imellem src og dst}$
\item[layering] et koncept der bruges til at adskille funktionalitet i
    net-stakken, sådan at der kun fokuseres på det vigtige på det givne lag.
    Der findes to modeller for lagdeling 5-lags og OSI 7-lags modellen.
\item[layer, application] øverste lag, hvor i applikationer såsom HTTP, SMTP og
    DNS er implementeret. applikations lagets pakker bliver ofte kaldet
    beskeder.
\item[layer, transport] ligger under applikations laget. Er her hvor TCP og UDP
    er implementeret. Ansvaret her er at transportere pakker fra ende til ende
    på applikation laget. En pakke på dette lag kaldes et segment.
\item[layer, network] ligger under transport laget. Er hjemsted for IP
    protokollen og routing protokoller generelt. Har til opgave at transportere
    segmenter fra en host til en anden. Pakker på dette lag kaldes datagrams.
\item[layer, link] ligger under netværks laget. Ansvaret her at at pakke
    datagrams, sådan at de kan overføres via en bestemt protokol, eksempler er:
    Ethernet, WiFi og Point-to-Point Protocol(PPP).  Ansvaret er ren af- og
    dekodning af datgrammer imellem to fysiske punkter(routere el.lign). Pakker
    på dette lag kaldes frames.
\item[layer, physical] ligger nederst i hierakiet. Ansvaret i det fysiske lag
    er at overføre bits fra den ene ende af et link til den anden ende. Hvor
    link laget står for at dekode et datagram sådan at det kan sendes via
    ethernet, er det fysiske lags ansvar at dekode datagrammets bits, sådan at
    de kan overføres via det ønskede medium, f.eks. twisted-pair kober, fiber
    optisk eller coaxial kabel.
\item[TCP] fork. Transmission Control Protocol. Er en forbindelses orienteret
    protocol med indbygget garanti for overlevering af pakker.
\item[TCP, RTT] fork. Round Trip Time. Se EstimatedRTT, DevRTT og TimeoutInterval,
    alle i forbindelse med TCP.
\item[TCP, estimatedRTT] = $(1-\alpha) \cdot EstimatedRTT + \alpha \cdot SampleRTT$,
    SampleRTT er et sample fra den nuværende forbindelse. $\alpha$ gives ofte
    værdien 0.125 aka 1/8
\item[TCP, devRTT] = $(1-B) \cdot DevRTT + B \cdot | SampleRTT - EstimatedRTT |$
\item[TCP, timeout interval] = $EstimatedRTT + 4 \cdot DevRTT$
\item[TCP, ACK] fork. Acknowledgment. Bruges fra en modtager af data til at
    signalere til afsenderen at data er modtaget.
\item[TCP, NAK] fork. Negative Acknowledgment. Bruges af en modtager til at
    signalere til afsenderen at der er sket en fejl i modtagelsen.
\item[TCP, checksum] en værdi der bruges til at kontrollerer om der er fejl i en
    pakke. Er også en del af UDP protokollen.
\item[TCP, Timer] implementeres ofte i protokoller med pålidig overførsel,
    sådan at pakker der enten ikke får et \textbf{ACK} tilbage, eller ikke
    ankommer(og dermed heller ikke bliver ACK'et), kan blive gensendt.
\item[TCP, Sequence number] et fortløbende nummer der identificerer pakker
    sådan at man kan nøjes med at gensende tabte pakker, og sådan at en
    datastrøm kan genskabes i rigtig rækkefølge hos modtageren.
\item[TCP, Window, pipelining] er en teknik der bruges for at udnytte linket
    bedre, i modsætning til stop-and-wait. Man tillader at flere pakker kan
    være ``in-flight'' og opbevarer så ikke ACK'ede pakker i en buffer, så de
    kan gensendes ved timeout el.lign.

\end{description}

\section{Eksempler}

\subsection{Time-Division Multiplexing}
Fil på 640.000 bits sendes fra Host A til Host B, alle mellemled bruger TDM med
24 slots og en bitrate på 1.536 Mbps og det tager 500 msec at etablerer en
forbindelse.

\[
\frac{1.536 Mbps}{24} = 64 kbps \\
\frac{640000 bits}{64 kbps} = 10 s
\]

Plus etablerings tiden, giver $10,5$ sekunder

\end{document}
