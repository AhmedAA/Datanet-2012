\section{Store and Forward}
\subsection{Processing and delay}
One of the main reasons for delay in packet switched networks is the output
buffer. If the output buffer becomes full, the router, or switch, begins to drop
the packets, resulting in a delay. This is also called ``packet loss''. Another
reason is store-and-forward, which requires the router, or switch, to have the
whole packet received before transmitting it. For large packets this would take
a lot of time.

\subsection{Transmission speed}
\textbf{Part 1}\\
To calculate the round trip time we add the nodal delays between each source.
From the laptop to the access point the nodal delay is 2 ms, from the access
point to the modem the delay is 1 ms, and finally from the modem to the DSLAM
the nodal delay is 5 ms. This gives 8 ms in total, and then we add the delay from the
DSLAM to the diku.dk server, which is 24 ms, resulting in a total trip time of
32 ms. Then we multiply by 2, since the round trip time is from the client to
the server, and back again, giving a round trip time of \textbf{64 ms}.

\noindent \textbf{Part 2}\\
According to the formula in the book (under transmission delay), we can
calculate the time it takes for the packet to be pushed onto to link, and sent
by the following $L/R$, where $L$ is the packet size in bits, and $R$ is the
link speed in bits per second (b/s). We have a wireless uplink on 54 Mb/s, speed
from the access point to the modem is 100 Mb/s, from modem to DSLAM is 2 Mb/s,
and from the DSLAM to DIKUs servers is 1 Gb/s.
This results in the following:
\[
\frac{5120000}{54 \cdot 10^6} + \frac{5120000}{100 \cdot 10^6} +
\frac{5120000}{2 \cdot 10^6} + \frac{5120000}{1 \cdot 10^7} = 2ms
\]
and then we have a RTT of 1.5 ms since she has to send a SYN, then receive a
SYN-ACK, and finally send an ACK, before she can send the data. We then end up
with a transmission time of $1.5ms + 2ms = 3.5 ms$
