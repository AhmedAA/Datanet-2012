\section{Buffers and latency}
\textbf{Part 1}\\
First, we recalculate the packet size to bits:
\[
512 \cdot 8 \cdot 10^6 = 4096000 bits
\]
Then we recalculate the link speed to bits per second:
\[
2.048 \cdot 1000 = 2048000 bits/sec
\]
We quickly see that $\frac{4096000}{2048000} = 2$, so the maximum buffer delay
would be 2 seconds.

\noindent \textbf{Part 2}
BDP tells you the maximum amount of data on a network circuit at any given time. This is important
for protocols; if the network has a very large end-to-end delay, then protocols relying on consistent
and rapid transmission will have issues. Instead, the protocols should take the network into consideration
and scale for a larger buffer transmission.

This means that if you have a buffer which is too small, you may end up
with insufficient amounts of data transferred across a large end-to-end
delay. Inversely, if your buffer is too large, then you will wait too long
to begin sending. As such, gauging the BDP is quite important when
considering buffer sizes.
