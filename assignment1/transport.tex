\section{Transport Protocols}
\subsection{TCP reliability and utilization}
\textbf{Part 1} \\
We will explain the need of this by listing what each step actually does, in the
three-way-handshake. The first SYN-packet from the client is a request to
synchronise a sequence number that handles packets (order of arrival, identify,
etc.). Next, the server sends a SYN-ACK packet, which accepts the synchronisation
request and gives a sequence number. Finally the ACK-packet from the client,
tells the server that the synchronisation has been accepted.

\noindent \textbf{Part 2} \\
TCP facilitates full-duplex traffic through the sequence numbers, between the
client and server. The traffic is fully controlled this way.

\subsection{Reliability vs overhead}
\textbf{Part 1} \\
TCP has various factors that create overhead. First of all there is the
three-way-handshake. There is also the assurance of packet delivery. If a
packet is dropped, TCP wants to send it again. Generally, UDP sends a stream of
data without any acknowledgement, whereas TCP wants to make sure that the data
has been received by the receiver. Many of the precautions taken by TCP, that
ensures a reliable and stable connnection, gives overhead.

\noindent \textbf{Part 2}
TCP connections are reliable in the sense that they establish a connection via
the three-way-handshake. In sense of the way TCP sends data, it makes sure that
lost packets are resent, and that they are received in the order they were sent,
making sure the receiver does not get incomprehensive data. Packets are always
delivered, unless someone gets completely disconnected from the internet.

\subsection{Use of transport protocols}
\textbf{DNS part 1} \\

\noindent \textbf{DNS part 2} \\

\subsection*{Simple HTTP requests and responses}
1. The minimum amount of packets that should be sent, is 5. We need to send the
standard three-way-handshake (three packets) with a payload from the last ACK,
then the response from the server, and a termination of the connection. \\

\noindent 2. If we assume the case from above, then we would have two packets
with data, out of the five packets in total. Since the last ACK packet, can have
data in its payload, and the server sends a response.

The order of the packets in HTTP matters, from a users point of view. It is not
intuitive for the user to receive the packets in random order, to see the website
get drawn from random positions in the browser.