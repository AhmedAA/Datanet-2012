\section{Transport Protocols}
\subsection{TCP reliability and utilization}
\textbf{Part 1} \\
We will explain the need of this by listing what each step actually does, in the
three-way-handshake. The first SYN-packet from the client is a request to
synchronise a sequence number that handles packets (order of arrival, identify,
etc.). Next, the server sends a SYN-ACK packet, which accepts the synchronisation
request and gives a sequence number. Finally the ACK-packet from the client,
tells the server that the synchronisation has been accepted.

\noindent \textbf{Part 2} \\
TCP facilitates full-duplex traffic through the sequence numbers, between the
client and server. The traffic is fully controlled this way.

\subsection{Reliability vs overhead}
\textbf{Part 1} \\
TCP has various factors that create overhead. First of all there is the
three-way-handshake. There is also the assurance of packet delivery. If a
packet is dropped, TCP wants to send it again.
