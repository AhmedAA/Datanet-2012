\section{Transport Protocols}
\subsection{TCP reliability and utilization}
\textbf{Part 1} \\
We will explain the need of this by listing what each step actually does, in the
three-way-handshake. The first SYN-packet from the client is a request to
synchronise a sequence number that handles packets (order of arrival, identify,
etc.). Next, the server sends a SYN-ACK packet, which accepts the synchronisation
request and gives a sequence number. Finally the ACK-packet from the client,
tells the server that the synchronisation has been accepted.

\noindent \textbf{Part 2} \\
TCP facilitates full-duplex traffic through the sequence numbers, between the
client and server. The traffic is fully controlled this way.

\subsection{Reliability vs overhead}
\textbf{Part 1} \\
TCP has various factors that create overhead. First of all there is the
three-way-handshake. There is also the assurance of packet delivery. If a
packet is dropped, TCP wants to send it again. Generally, UDP sends a stream of
data without any acknowledgement, whereas TCP wants to make sure that the data
has been received by the receiver. Many of the precautions taken by TCP, that
ensures a reliable and stable connnection, gives overhead.

\noindent \textbf{Part 2}
TCP connections are reliable in the sense that they establish a connection via
the three-way-handshake. In sense of the way TCP sends data, it makes sure that
lost packets are resent, and that they are received in the order they were sent,
making sure the receiver does not get incomprehensive data. Packets are always
delivered, unless someone gets completely disconnected from the internet.

\subsection{Use of transport protocols}
\textbf{DNS part 1} \\
DNS employs the UDP protocol. DNS has a need to be extremely fast, as it is a frequently used service.
The overhead of setting up and tearing down TCP conections, and the need to maintain stateful information
about a TCP connection until it is torn down, are some of the reasons TCP is unsuited. In addition, the
DNS Message format is structured in a way that attempts to give as much information as possible in a single
request, so that no additional communication to the same name-server should be required.

\noindent \textbf{DNS part 2} \\
TCP's 3-way handshake would provide the form of end-to-end validation that DNSSEC f.ex. provides, preventing
a range of DNS cache poisoning attacks from working. 

If a DNS server of an ISP is poisoned, an attacker can re-direct to their own A records, that to the end-user will
look identical. In a hypothetical scenario, consider a spoofed gmail.com A record, re-directing to a page that looks
an exact replica of http://www.gmail.com/. Users would input their username and password, and if the attacker were smart,
he would re-direct that input to the real www.gmail.com and the user would never know they had given away their username
and password. The same situation could apply to something like a net-login for banking, or a variety of other things.

\subsection*{Simple HTTP requests and responses}
1. The minimum amount of packets that should be sent, is 5. We need to send the
standard three-way-handshake (three packets) with a payload from the last ACK,
then the response from the server, and a termination of the connection. \\

\noindent 2. If we assume the case from above, then we would have two packets
with data, out of the five packets in total. Since the last ACK packet, can have
data in its payload, and the server sends a response.

The order of the packets in HTTP matters, from a users point of view. It is not
intuitive for the user to receive the packets in random order, to see the website
get drawn from random positions in the browser. \\

\noindent 3. UDP could probably be used for HTTP, but it certainly would not
give the desired effect. UDP does not guarantee delivery of packets, nor does it
reassemble the packets in the correct order. Thus a user could load a HTTP page
without getting all elements on the page, or the elements would load in random
order since we can not assure the correct order the packets should be loaded in.