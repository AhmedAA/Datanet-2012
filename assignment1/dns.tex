\subsection{DNS}
\subsubsection{DNS Provisions}
\textbf{Effeciency} is, amongst other things, achieved in the DNS Message Format. It allows the serving of multiple RRs 
in a single request, meaning that any DNS request will either lead you to the target you're looking for (A/CNAME/MX), or
provide you with further NS records to ask another DNS server for the answer. In addition, \textbf{cacheing} is an important
strategy in maintaining the effeciency of DNS requests. Requests will drive you toward the correct DNS servers that
houses the information you need, hopefully localized near the client. Fault tolerance and scalability are provided by IP-reordering
on each new request, along with the ability to forward to a number of different available DNS servers to serve the request.

\subsubsection{DNS Lookup and Format}
\textbf{Part 1:}\\
CNAME records allow you to provide any number of alias hostnames, allowing you to have f.ex ftp.myserver.com and webmail.myserver.com
that both point to the same IP address. DNS provides simple load-balancing through re-ordering the IP addresses that it returns, on each
subsequent request. This is a primitive form of round-robin, and will in most cases mean that a different server is contacted by the requester,
since most services will simply pick the first IP address provided. 

\noindent \textbf{Part 2:}\\

\noindent \textbf{Part 3:}\\


\subsubsection{}