\subsection{HTTP}

\subsubsection{HTTP Semantics}
\paragraph{Part 1} The purpose of the method field in HTTP is to give different
options as to what is being done. Am I Sending you some content or just trying
to get a resource. The practical difference between a GET and POST request is
almost invisible - although recent developments with webservices have
emphasized the difference, and made a clear stand towards the no-side-effect of
GET requests as opposed to POST, that isn't considered a ``safe'' method.

\paragraph{Part 2} Without the Host-header it is not possible, to host multiple
domains with different content on a single IP. Said in another way, the
Host-header enables a single host on a single IP to identify several logical
hosts and by that multiple ``websites''.

\paragraph{Part 3} The returned status code is 302, which means the requested
resource was found at this location, but has moved temporarily. The redirect
location is http://www.google.dk(Location-header), Cache-control indicates that
the response is not cached(other than the final receiver) aka private, 2 
cookies are included and google has their own(so far) unknown server called 
``gws''. The responsecode in this case is important in 2-ways, first it tells
us that google would like us to use the danish google domain, secondly it is
only a temporary redirect, meaning google could move their danish customers to
f.ex. sweden with minor problems.

\subsubsection{HTTP Headers and fingerprinting}
\paragraph{Part 1} These header-fields are used to introduce state in the
otherwise stateless HTTP - therefore they must also be able to identify the
user uniqely, such that UserA doesn't get UserB shoppingcart shown.

\paragraph{Part 2} Security and sensitive information..

\paragraph{Part 3} E-TAGS See section 13.3.2-3 and 14.19 RFC2616

\subsubsection{The case of Deep Packet Inspection}
\paragraph{Part 1} To block grooveshark.com, '3' could have used IP or DNS, or
as described below a web-proxy. IP blocking, DNS blocking 

\paragraph{Part 2} This is a violation since a lower layer service is made
aware of a higher level service. To enforce the layering, '3' could force their
customers to use a HTTP-Proxy which they control, the proxy could then be used 
to control what traffic is allowed and not.
