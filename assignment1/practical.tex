\section{Socket Programming}

\subsection{Question 1}
The peer client needs 2 sockets, a client socket connected to the name server,
and a client socket in listen mode. One could argue that a third socket for
client input is also needed, but since this isn't a socket that needs to be
opened or treated in any way it is not counted in.

\subsection{Question 2}
The choice to use a buffer size, limits the message size, this could be
prevented by using a special token, on the other hand a special token would
prevent users from using this token in their messages. So the ideal solution
would be to use a combination of tokens, this could be ``CRLF;CRLF;'', this
would allow people to use a single new line, but not multiple.

\subsection{Question 3}
The answer is both yes and no! and it could cause some trouble. If clientA
connects through the local loopback interface(127.0.0.1) the server would
then serve this address for client B, which in turn then connects to itself for
chatting with client A. On the other hand if client A connects through the
given IP(10.0.0.1) then chatting is possible(of course without taking firewalls
and the like, into consideration)

Simply prefixed by CA, CB or NS respectively ClientA, ClientB or NameServer.
Asserting the following IP's, NS: 10.0.0.1, CA: 10.0.0.1, CB: 10.0.0.2

\begin{verbatim}
CA -> NS: HELLO clienta 1234
NS -> CA: 100 CONNECTED

CB -> NS: HELLO clientb 1234
NS -> CB: 100 CONNECTED

CB -> NS: LOOKUP clienta
NS -> CB: 400 INFO <ip> 1234

CB -> CA: HELLO clientb 1234
CA -> CB: 200 CONNECTED

CB -> CA: MSG clientb Hello there

CA -> CB: MSG clienta Well, hello to you
\end{verbatim}

\subsection{Question 4}
\subsection{Question 5}
\subsection{Question 6}
\subsection{Question 7}
\subsection{Question 8}
\subsection{Question 9}
\subsection{Question 10}
