\section{Socket Programming}

\subsection{Question 1}
The peer client needs 2 sockets, a client socket connected to the name server,
and a client socket in listen mode. One could argue that a third socket for
client input is also needed, but since this isn't a socket that needs to be
opened or treated in any way it is not counted in.

\subsection{Question 2}
The choice to use a buffer size, limits the message size, this could be
prevented by using a special token, on the other hand a special token would
prevent users from using this token in their messages. So the ideal solution
would be to use a combination of tokens, this could be ``CRLF;CRLF;'', this
would allow people to use a single new line, but not multiple.

\subsection{Question 3}
The answer is both yes and no! and it could cause some trouble. If clientA
connects through the local loopback interface(127.0.0.1) the server would
then serve this address for client B, which in turn then connects to itself for
chatting with client A. On the other hand if client A connects through the
given IP(10.0.0.1) then chatting is possible(of course without taking firewalls
and the like, into consideration)

Simply prefixed by CA, CB or NS respectively ClientA, ClientB or NameServer.
Asserting the following IP's, NS: 10.0.0.1, CA: 10.0.0.1, CB: 10.0.0.2

\begin{verbatim}
CA -> NS: HELLO clienta 1234
NS -> CA: 100 CONNECTED

CB -> NS: HELLO clientb 1234
NS -> CB: 100 CONNECTED

CB -> NS: LOOKUP clienta
NS -> CB: 400 INFO <ip> 1234

CB -> CA: HELLO clientb 1234
CA -> CB: 200 CONNECTED

CB -> CA: MSG clientb Hello there

CA -> CB: MSG clienta Well, hello to you
\end{verbatim}

\subsection{Question 4}
Peers are either 'connected' or not 'connected', based on their
ability to reach the nameserver. As such, if the socket to the nameserver
dies, all communication must be halted until the connection is re-stablished.

The alternative, is to disconnect entirely if the connection to the nameserver fails.

Socket connections to peers need not influence the rest of the flow, they should simply die
off and disconnect. It may be user-friendly in those cases to try and reconnect, if possible.
Often, popular services provide a reconnection-period, where they will try and re-establish the
connection behind-the-scenes. However, that is beyond the scope of what we've had the time to do.


\subsection{Question 5}
There are multiple ways to solve this. The reason we need to contact N peers, is that we're doing all
of the work ourselves. In order to reduce this, the work must be farmed out to other peers, which will
allow us to achieve an O(log n) running-time, if we're using a data-structure that maintains a log n height.

The basic idea is to distribute the broadcast amongst the peers. A naive implementation there-of might work
through a simple tree-structure, that is passed on to child peers, which will pick off the top child and pass
the left and right side off to two other peers. Other implementations utilize minimum spanning trees and reverse
shortest-path lookups, to make sure that the same broadcast message isn't propagated across the same node more than
once. Consider a line of nodes, only connecting to the node ahead and behind of it, and how many messages need to pass
through them if there are say, 4 in a line, and the last one connects to 20 nodes.

\subsection{Question 6}
This depends on the amount of joins and leaves. In a system, where we have a high frequency of joins and leaves, it is
definitely advisable to keep the sockets alive and well. The overhead of re-establishing the connection may exceed the
cost of keeping it alive, when X amount of peers connect/leave every period Y.

\subsection{Question 7}
One option is to store and maintain the notion of 'groups' on the server, and expanding the protocol to provide access to
a command to form a group, invite people to your group and leave a group. This maintains the information entirely on the server.

Another option is to simply let clients say 'I want to broadcast this message to these 5 nicks'. This has the advantage of not
requiring the server to maintain any state to support this, and only a single addition to the protocol, namely a 'Broadcast to these
select nicks' command. This would have the downside, that you need to make sure the user has permissions to broadcast to those individuals,
requiring either the nicks to authenticate against the server with a new protocol command designed for 'Can this user broadcast to you?'.

As such, the burden of deciding whether someone can message you, remains on the client.

\subsection{Question 8}
Yes. The connection is not encrypted in any way. A simple solution is to utilize SSL to encrypt communication, such that you have the
benefits of the security provided by SSL. 

\subsection{Question 9}
Our implementation is not robust, no. However, a robust implementation would need to verify the syntax and format of the messages being
sent, to prevent malicious peers from abusing things. Also, proper timeouts will help to eliminate hanging connections and similar.

\subsection{Question 10}

