\section{Internet Protocol}
This section answers the questions regarding IP(Section 1.2)

\subsection{Addresses and network masks}
\paragraph{Part 1: } Se below listing for answers o the different questions.

\begin{enumerate}
\item From a routers Point of view, network masks minimizes the information
needed to route efficiently. Said differently, A route can based on an IP'
prefix tell where to send the information.
\item Network masks may bee expressed as an IP adress, that meaning in
dot-notation, formed as 4 numbers expressing the number of bits occupied by the
prefix. 255.225.255 is not a valid netmask, since it translates to
11111111.11100001.11111111.0. The slash-notation just specifies how many of the
bits is prefix in the mask.
\item In the classfull scheme, network masks are always divided in 8-bit and
always occupying a full 8-bit group, this results in 4 different classes, A, B,
C and D, occupying respectively 0, 8, 16 or 32 bits of the network address.
Classles on the other hand only specifies the number of bits occupied and are
therfore not bound by the quads.
\item Yhe networkaddress part can be derived by subtracting the inverted
network mask from the given address. When presented with an address and the 
mask, you can derive the broadcast address by inverting the networkmask and 
adding the network address. The number of available addresses is: 
$2^{32 - n}-1$.
\end{enumerate}

\subsection{Network Address Translation}

