\section{Internet Protocol}
This section answers the questions regarding IP(Section 1.2)

\subsection{Addresses and network masks}
\textbf{Part 1: } See below listing for answerst o the different questions.

\begin{enumerate}
\item From a routers PoV, a network mask minimizes information needed to route properly.
    IPv4 addresses are divided into two parts, the network prefix and host identifier.
    All hosts on a subnet have the same network prefix, thus allowing the selective routing
    of IP packets via routers.
\item Network masks may bee expressed as an IP adress, that meaning in
    dot-notation, formed as 4 numbers expressing the number of bits occupied by the
    prefix. 255.225.255 is not a valid netmask, since it translates to
    11111111.11100001.11111111.0. The slash-notation just specifies how many of the
    bits is prefix in the mask.
\item In the classfull scheme, network masks are always divided in 8-bit and
    always occupying a full 8-bit group, this results in 4 different classes, A, B,
    C and D, occupying respectively 0, 8, 16 or 32 bits of the network address.
    Classles on the other hand only specifies the number of bits occupied and are
    therfore not bound by the quads.
\item The networkaddress part can be derived by subtracting the inverted
    network mask from the given address. When presented with an address and the 
    mask, you can derive the broadcast address by inverting the networkmask and 
    adding the network address. The number of available addresses is: 
    $2^{32 - n}-1$.
\end{enumerate}

\textbf{Part 2: } Network overview:

% NW Size, NW addr, mask, broadcast, first, 5th, last
\scalebox{0.9}{
\begin{tabular}{| r | l | l | l | l | l | l |}
\hline
Size  & NW addr        & mask            & broadcast       & first          & 5th & last \\ \hline
31    & 130.225.165.16 & 255.255.255.224 & 130.225.165.031 & 130.225.165.01 & 130.225.165.1 & 130.225.165.30 \\ \hline
511   & 10.0.42.0 & 255.255.254.000 & 010.000.042.255 & 010.000.042.01 & 010.000.042.6 & 010.000.043.254 \\ \hline
1     & 4.2.2.1 & 255.255.255.255 & 255.255.255.255 & 004.002.002.01 & N/A & N/A \\ \hline
16383 & 192.38.108.0 & 255.255.192.000 & 192.038.127.255 & 192.038.064.01 & 192.038.064.6 & 192.038.127.254 \\ \hline
\end{tabular}
}

\subsection{Network Address Translation}
\textbf{Part 1: } A NAT-enabled router may keep a table from Internal-IP to
external-IP and port, also keeping track of some identifier for each of the
hosts, in the event of multiple internal hosts connecting to the same external
host.

\textbf{Part 2: } The simple scenario where you have 33000 connections
outbound, will result in exhausting the table for space, this should be to the
same host.

\textbf{Part 3: } For NATing to be succesfull it must inspect and transform the
information stored in the ``deeper'' levels of an IP packet, therefore going to
fx. TCP/UDP layers, or even worse, when a higher level protocol is network
dependent, this could result in NAT breaking the protocol.

