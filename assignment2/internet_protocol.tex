\section{Internet Protocol}
This section answers the questions regarding IP(Section 1.2)

\subsection{Addresses and network masks}
\textbf{Part 1: } See below listing for answerst o the different questions.

\begin{enumerate}
\item From a routers PoV, a network mask minimizes information needed to route properly.
    IPv4 addresses are divided into two parts, the network prefix and host identifier.
    All hosts on a subnet have the same network prefix, thus allowing the selective routing
    of IP packets via routers.
\item A network mask can be split into 4 8-bit sections, each representing the
    number of bits used by the network mask. This is equivalent to 4 decimal
    numbers from 0-255, expressed in dot-notation as if it were an IP address.
    The network mask 255.225.255.0 (binary form 11111111.11100001.11111111.0)
    is not valid. In a network mask, the left-most part must be all 1's, and
    the right-most part all 0's, which is clearly not the case here. The 
    slash-notation /28 specifies how many of the 32 available bits are prefix.
    As such, /28 is equal to the network mask 255.255.255.240.
\item In the classfull scheme, network masks are always divided in 8-bit and
    always occupying a full 8-bit group, this results in 4 different classes, A, B,
    C and D, occupying respectively 0, 8, 16 or 32 bits of the network address.
    Classles on the other hand only specifies the number of bits occupied and are
    therfore not bound by the quads.
\item The networkaddress part can be derived by subtracting the inverted
    network mask from the given address. When presented with an address and the 
    mask, you can derive the broadcast address by inverting the networkmask and 
    performing a Binary OR between the two. 
    The number of available addresses is: $2^{n}-2$ where n is the number
    of bits used for the host portion of the address.
\end{enumerate}

\textbf{Part 2: } Network overview:

% NW Size, NW addr, mask, broadcast, first, 5th, last
\scalebox{0.9}{
\begin{tabular}{| r | l | l | l | l | l | l |}
\hline
Size  & NW addr        & mask            & broadcast       & first          & 5th & last \\ \hline
31    & 130.225.165.16 & 255.255.255.224 & 130.225.165.31 & 130.225.165.1 & 130.225.165.5 & 130.225.165.30 \\ \hline
511   & 10.0.42.0 & 255.255.254.0 & 10.0.42.255 & 10.0.42.1 & 10.0.42.5 & 10.0.43.254 \\ \hline
1     & 4.2.2.1 & 255.255.255.255 & 4.2.2.1 & 4.2.2.1 & N/A & 4.2.2.1 \\ \hline
16383 & 192.38.108.0 & 255.255.192.0 & 192.38.127.255 & 192.38.108.1 & 192.38.108.5 & 192.38.127.254 \\ \hline
\end{tabular}
}

\subsection{Network Address Translation}
\textbf{Part 1: } A NAT-enabled router may keep a table from Internal-IP to
external-IP and port, also keeping track of some identifier for each of the
hosts, in the event of multiple internal hosts connecting to the same external
host.

\textbf{Part 2: } The simple scenario where you have 33000 connections
outbound, will result in exhausting the table for space, this should be to the
same host.

\textbf{Part 3: } For NATing to be succesfull it must inspect and transform the
information stored in the ``deeper'' levels of an IP packet, therefore going to
fx. TCP/UDP layers, or even worse, when a higher level protocol is network
dependent, this could result in NAT breaking the protocol.
