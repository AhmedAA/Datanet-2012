\section{Practical part}
This section gives answers to the questions asked regarding the DHT
implementation of the name server.

\subsection{Question 1}
It is possible to use the same socket for both protocols. To
differentiate between the two protocols, we could use some exception
handling that checks the format of the request to determine which
protocol should be used.

It would be easier to implement two different sockets, but doing so
will not result in a faster service. It would just move the
calculation overhead, from the exception handling, to the sockets,
resulting in more space being used. Again, it would be easier to
implement, but no speed-up is being achieved.

\subsection{Question 2}

\subsection{Question 3}
It is possible to broadcast a message through the whole DHT. This is
done by sending a message to the sending nodes neighbours, and letting
the neighbours forward that message. This a naive implementation.

\subsection{Question 4}
It should be stressed that the same value for $k$ should be used, no
matter what version is being used. If different values are used, a
peer might think she is contacting a person she knows, although it
could be an entirely different one, due to the different values of $k$.

\subsection{Question 5}
Due to ambiguity it should not be able to send a message to nodes with
conflicting ids, but the service should still be able to function as a
whole. It can be avoided by checking all $ks$ in the $kbuckets$, and
make sure that the username a user wants, is not already in use.

The assumption in the question guarentees that this will not happen.

\subsection{Question 6}
A user might disconnect from the DHT in a non-graceful way, which
could result in the peer still hanging in the DHT.

\subsection{Question 7}

\subsection{Question 8}

\subsection{Question 9}

\subsection{Question 10}