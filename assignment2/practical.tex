\section{Practical part}
This section gives answers to the questions asked regarding the DHT
implementation of the name server.

\subsection{Question 1}
It is possible to use the same socket for both protocols. To
differentiate between the two protocols, we could use some exception
handling that checks the format of the request to determine which
protocol should be used.

It would be easier to implement two different sockets, but doing so
will not result in a faster service. It would just move the
calculation overhead, from the exception handling, to the sockets,
resulting in more space being used. Again, it would be easier to
implement, but no speed-up is being achieved.

\subsection{Question 2}
Utilizing pickle to serialize and de-serialize, combined with messages packed
as Python objects, would be our immediate way to implement things. As we
haven't actually done the programming, we can't speak on what we actually
did, as we haven't :)

Utilizing a packed object, we would need a type and potentially a dictionary
of arguments, that would be understood differently depending on the type of
the message.

\subsection{Question 3}
It is possible to broadcast a message through the whole DHT. This is
done by sending a message to the sending nodes neighbours, and letting
the neighbours forward that message. This a naive implementation.

\subsection{Question 4}
It should be stressed that the same value for $k$ should be used, no
matter what version is being used. If different values are used, a
peer might think she is contacting a person she knows, although it
could be an entirely different one, due to the different values of $k$.

\subsection{Question 5}
Due to ambiguity it should not be able to send a message to nodes with
conflicting ids, but the service should still be able to function as a
whole. It can be avoided by checking all $ks$ in the $kbuckets$, and
make sure that the username a user wants, is not already in use.

The assumption in the question guarentees that this will not happen.

\subsection{Question 6}
A user might disconnect from the DHT in a non-graceful way, which
could result in the peer still hanging in the DHT.

\subsection{Question 7}

\subsection{Question 8}
A sybil attack is an attack that creates pseudonymous entities, using them
to skewer the reputation system put in place by a peer-to-peer network.
Identities are fairly easy to establish in this setup, and as such the existing
setup is extremely vulnerable to a Sybil attack. One way to alleviate part of
this is to establish a chain of trust within the peer-to-peer network, where
trust is gradually stepped up based on how many other peers trust a given peer.

One practical solution is a centralized (or de-centralized) lookup mechanism
amongst either trusted peers or a central name lookup service, every time a
peer needs to be validated as trusted / non-trusted. The notion of super-peers
or trust-certifiers may enter the system at this point.
\subsection{Question 9}

\subsection{Question 10}
\subsubsection{Advantages of DHT}
\begin{enumerate}
    \item No single point of failure.
    \item Harder to trace the origin of file downloads
    \item Is most likely much more local, than the central nameserver. Consider
        a Danish client connecting to an American nameserver, and the latency
        that automatically induces.
    \item In extension of 'no single point of failure', this applies both in
        the network aspect as well as file-retention aspect. If a single node
        goes down, there is a high likelyhood that another node will have the
        same file(s) someone is looking for. As such, it is much, much harder
        to 'eliminate' a file or group of files from the network.
\end{enumerate}
\subsection{Disadvantages of DHT}
\begin{enumerate}
    \item Making sure the DHT is properly kept up-to-date requires additional
        book-keeping and protocol additions for this purpose alone.
    \item Attacks that target distributed networks of peers are vulnerable to
        the missing central authentication, unless such is implemented. See
        sybil attack above.
    \item It may potentially be slower to find files in a distributed setup,
        than if there were a central file authority.
\end{enumerate}
