\section{Routing}
The following tables describes the routing table for
node E from propagation 0 to 2.

\subsection{Initial routing table of E T=0}
\begin{tabular}{| l | l | l |}
    \hline
    Destination & Cost & Next Hop \\ \hline
    A & \cellcolor{red}$\infty$ & -- \\ \hline
    B & \cellcolor{red}$\infty$ & -- \\ \hline
    C & 11 & C \\ \hline
    D & 17 & D \\ \hline
\end{tabular}
\subsection{E after T=1}
\begin{tabular}{| l | l | l |}
    \hline
    Destination & Cost & Next Hop \\ \hline
    A & 32 & D \\ \hline
    B & 29 & D \\ \hline
    C & 11 & C \\ \hline
    D & 16 & C \\ \hline
\end{tabular}
\subsection{E after T=2}
\begin{tabular}{| l | l | l |}
    \hline
    Destination & Cost & Next Hop \\ \hline
    A & 31 & C \\ \hline
    B & 28 & C \\ \hline
    C & 11 & C \\ \hline
    D & 16 & C \\ \hline
\end{tabular}
\subsection{Failure situation}
There is a problem often called the 'count-to-infinity' problem,
which can occur in a DV algorithm utilizing f.ex. Bellman-Ford. The
basic premise of the issue, is that if a link cost increases as a
result of a link failure, then you risk a situation where two nodes
will continually update the other node with a new cost, until a
threshhold is met where one of the nodes has a less costly route
elsewhere. This can happen when a node falsely believes that it can
use a lower-cost route, through another node, which itself believes
that it can use a lower-cost route through the first node. As such,
a ping-pong effect occurs until the cost is sufficiently high.

There are a number of solutions to this problem. One such attempt at
a solution is poisoned reverse, which attempts to solve the problem by
letting nodes tell their next-hop that their count to the destination
they're headed towards is infinite. This prevents routin back through
the same node, but does not work in a situation where 3 or more nodes
are involved in the infinite count. Poisoned reverse also suffers from
the issue that it significantly increases routing announcement sizes,
for a variety of network topologies.

Another solution is split-horizon with poison reverse. Split-horizon
dictates that a router is prohibited from advertising a route back
onto the interface from which it picked it up. Notably, RIP uses
split-horizon with poisoned reverse.


